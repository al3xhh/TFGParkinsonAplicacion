\documentclass[11pt,spanish]{article}

%% Language and font encodings
\usepackage[utf8x]{inputenc}
\usepackage[T1]{fontenc}
\usepackage[spanish,activeacute]{babel}

%% Sets page size and margins
\usepackage[a4paper,top=3cm,bottom=2cm,left=3cm,right=3cm,marginparwidth=1.75cm]{geometry}
\usepackage{indentfirst}

%% Useful packages
\usepackage{amsmath}
\usepackage{graphicx}
\usepackage[colorinlistoftodos]{todonotes}
\usepackage[]{hyperref}
\usepackage{wrapfig}
\usepackage{float}

\definecolor{gris}{RGB}{220,220,220}

\author{
\Large Christian González García-Muñoz \\ 
\Large Alejandro Huertas Herrero 
}
\date{}
\begin{document}
\begin{titlepage}
	\centering
    \includegraphics[width=8cm]{ucm-logo.png}
    \vskip 1cm
    
    \centering
    {\huge Reconocimiento de actividades domésticas en personas con enfermedades neurodegenerativas mediante una pulsera de sensores 
    }
    \newline
    \newline
    \newline
    \newline
    
	\centering \large { Christian González García-Muñoz \\  Alejandro Huertas Herrero \\ 
    					\bigskip Dirigido por: Javier Arroyo \\ Codirigido por: Marlon Cárdenas \\ \bigskip}
    \vskip 1cm
    \centering \Large { Trabajo de fin de grado del Grado en Ingeniería Informática }
	\vskip 1cm
    \centering \large {Facultad de Informática \\ Universidad Complutense de Madrid \\ Curso 2017-2018}
    \vskip 0.5cm
    \centering \large {21 Mayo de 2018}

\end{titlepage}
\clearpage
\vphantom{a}
\newpage

\addtocontents{toc}{\protect\setcounter{tocdepth}{0}}

\section*{Resumen}
\addcontentsline{toc}{section}{Resumen}
El objetivo del proyecto es implementar una aplicación Android y toda la infraestructura necesaria para dar soporte a las funcionalidades de esta. La aplicación facilita tanto el día a día del paciente como el diagnostico de la enfermedad por parte del medico. En concreto, la aplicación permite al paciente (o al encargado del cuidado del mismo) añadir recordatorios con los días y horarios en los que debe tomar cada pastilla y tener un diario de las actividades que el paciente realiza como parte de su rutina de ejercicios, ya sea esta propia o recomendada por un medico. Todas estas acciones son monitorizadas por sensores BLE (Bluetooth de baja energía) conectados a la aplicación para después ser procesados y representados de forma amigable para el medico que deba interpretarlos. 
\newline

La aplicación Android se complementa con una aplicación web sencilla e intuitiva que permite proporcionar datos al médico. En ella, se pueden visualizar mediante gráficos la evolución de los datos recogidos por los sensores de cada paciente.
\newline

Tanto la aplicación Android como la aplicación web hacen uso de un API REST para enviar o recibir datos del back-end.
\newline

Como se ha dicho anteriormente, la aplicación permite monitorizar al paciente mediante sensores BLE. La aplicación soporta dos tipos de sensores, por un lado tenemos un sensor más “rudimentario” y por otro lado un sensor mas amigable con el usuario que lo lleve, con forma de pulsera y con una pantalla incorporada. Ambos sensores cuentan a su vez con varios servicios, de los cuales los utilizados para este proyecto han sido los acelerómetros y los giroscopios, ambos de tres ejes. La integración de la aplicación Android con los sensores BLE ha sido una de las mayores dificultades que nos hemos encontrado a la hora de llevar a cabo el proyecto debido a la compleja interfaz que ofrece Android para trabajar con este tipo de sensores.
\newline

Resumiendo, las tecnologías utilizadas han sido las siguientes:

\begin{itemize}
    \item Android para el desarrollo de la aplicación. 
	\item SQLite para almacenar los datos dentro del teléfono. 
	\item Bluetooth Low Energy para la comunicación con los sensores. 
	\item Python para procesar los datos y para implementar el API REST
    \item PHP, HTML5 y Bootstrap para el desarrollo de la web que permite visualizar los datos.
	\item MongoDB para almacenar en el back-end todos los datos recogidos.
    \item Apache como servidor web para hostear la aplicación web.
    \item Docker para desplegar la infraestructura necesaria de forma automática (API REST, servidor web y MongoDB).  
    \newline
\end{itemize}

{\bf Palabras clave:} aplicación Android, teléfono inteligente, Parkinson, bluetooth, Internet de las cosas, sensores, movimiento, actividades, medicamentos y visualización.
\newpage

\section*{Abstract}
\addcontentsline{toc}{section}{Abstract}
The aim of this work is to develop an Android application that helps people with Parkinson's disease. Mainly, it is focused in pills reminders and it allows have a history of all the activities that patient makes in its routine. All these actions are recorded by the bluetooth sensors which can be connected to the application. The application uses a server to store and process all the data.
\newline

In order to complement the work and provide data to the doctor or the patient supervisor, it has been developed an easy and intuitive web application. Data collected by the sensors can be visualized by some graphics.
\newline

As it has been said, the application conects with bluetooth sensors. There are two sensor types. On the one hand, we have a rudimentary one. On the other hand, we have a sensor with a bracelet and screen included. Both of them have some services, we have used three axis accelerometer and three axis gyroscope. One more thing to highlight is that both sensores are low energy bluetooth (BLE) which make the connection with the application more complex, because they easily lose the connectivity.
\newline

Summarizing, we have used the following technologies:

\begin{itemize}
    \item Android to develop the application. 
	\item SQLite to store data inside the smartphone. 
	\item Bluetooth for the communication with the sensors. 
	\item Python to develop the server which stores and proccess the data (Eve framework)
    \item PHP, HTML5 y Bootstrap to develop the web application.
	\item Mongodb to store the data in the server.
    \newline
\end{itemize}
\newpage

\addtocontents{toc}{\protect\setcounter{tocdepth}{3}}

\newpage
\tableofcontents
\newpage
\listoffigures
\newpage

\section{Estado del arte}
“La enfermedad de Parkinson (EP), también denominada mal de Parkinson, parkinsonismo idiopático, parálisis agitante o simplemente párkinson, es una enfermedad neurodegenerativa crónica caracterizada por bradicinesia (movimiento lento), rigidez (aumento del tono muscular) y temblor.” - Wikipedia
\newline

Gran parte de los síntomas de la enfermedad son motores, por eso decidimos hacer una aplicación basada en la utilización de sensores. Se han hecho numerosos estudios los cuales utilizan esta tecnología y a su vez utilizan los datos recogidos para aplicar técnicas de análisis de datos como machine learning sobre ellos.
\newline

Para introducirnos en el estado del arte, leímos numerosos artículos científicos, en concreto los más útiles e interesantes, son los que se listan a continuación:
\newline

\begin{itemize}
	\item Quantitative wearable sensors for objective assessment of Parkinson’s diseases.
	\item Analysis of feature space for monitoring persons with Parkinson’s disease with application to a wireless wearable sensor system. 
	\item Detection of walking periods and number of steps in older adults and patients with Parkinson’s disease: accuracy of a pedometer and an accelerometry-based method. 
	\item Monitoring Motor Fluctuations in Patients With Parkinson’s Disease Using Wearable Sensors . 
	\item Feasibility and effects of home-based smartphone-delivered automated feedback training for gait in people with Parkinson’s disease: A pilot randomized controlled trial.
	\item An accelerometry-based study of lower and upper limb tremor in Parkinson’s disease. 
	\item Detecting and monitoring the symptoms of Parkinson’s disease using smartphones: A pilot study.
	\item A smartphone-base architecture to detect and quantify freezing of gait in parkinson’s disease.
    \newline
\end{itemize}

Todos ellos tienen como nexo común, la utilización de sistemas basados en sensores para analizar las actividades que realizan los pacientes. La mayoría de ellos utilizan cuadros de actividades fijados previamente. 
\newline

Nuestra aplicación está diseñada para poder usarla para monitorizar al paciente en un entorno libre, en su día a día, o mientras realiza una tabla de ejercicios previamente definida, en este caso nuestra aplicación permite cruzar los datos de monitorización obtenidos con los horarios de la toma de la medicación, de esta forma se podría ver el efecto que tienen los medicamentos sobre cada paciente.
\newpage

\section{Aplicación Android}
\subsection{Descripción}
La aplicación ha sido desarrollada para dispositivos Android, con una versión igual o superior a Android 4.4 (KitKat) Cuenta con un diseño sencillo e intuitivo, ya que está pensada para personas que pueden tener temblores mientras la usen o pueden encontrarse en una avanzada edad. Toda la recogida de datos de los sensores está automatizada, el paciente sólo tiene que conectar la aplicación con cada sensor y puede olvidarse del teléfono en ese momento. Se ha intentado que la aplicación sea lo menos invasiva posible, para ello tan sólo cuenta con notificaciones cuando se tiene que tomar las pastillas. Cada notificación indica la pastilla que debe tomarse y cuenta con dos opciones, de esta forma se puede llevar un control de las pastillas que se ha tomado (esto quedaría pendiente como trabajo futuro)
\newline

\subsection{Capturas}
\subsubsection{Menú inicio}
\begin{figure}[!htb]
\centering
\includegraphics[width=0.25\textwidth]{Inicio.jpg}
\caption{Pantalla principal de la aplicación.}
\end{figure}
Esta es la pantalla que aparece cuando se inicia la aplicación. Podemos ver dos grandes botones, el primero de ellos permite acceder a todo lo relacionado con las actividades y el segundo a todo lo relacionado con los medicamentos.
\newline

Puede verse como en el menú superior hay dos botones. El botón de enviar, permite enviar todos los datos al servidor. El botón emparejar permite acceder al menú dónde se configuran y se emparejan los sensores con la aplicación.
\newpage

\subsubsection{Actividades}
\begin{figure}[!htb]
\minipage{0.32\textwidth}
  \includegraphics[width=\linewidth]{Actividades.jpg}
  \caption{Actividades}
\endminipage\hfill
\minipage{0.32\textwidth}
  \includegraphics[width=\linewidth]{AnyadirActividad.jpg}
  \caption{Añadir actividad}
\endminipage\hfill
\minipage{0.32\textwidth}%
  \includegraphics[width=\linewidth]{MenuActividades.jpg}
  \caption{Menú actividad}
\endminipage
\end{figure}

Las tres pantallas de arriba muestran cómo se trabaja con las actividades. La pantalla principal cuenta con una lista de todas las actividades que se realizan, a modo de histórico y con un botón que permite añadir una nueva actividad al sistema.
\newline

La información que se guarda de cada actividad es:

\begin{itemize}
	\item {\bf Nombre} de la actividad.
	\item {\bf Duración} de la actividad.
	\item {\bf Hora} a la que se empezó a realizar la actividad.
	\item Cualquier {\bf observación} que pueda hacer que los datos recogidos por los sensores en esa actividad no sean válidos.
\end{itemize}

La actividad puede ser añadida antes o después de realizarla, lo importante es que mientras se realice los sensores están conectados y recogiendo datos.
\newline

Al pulsar en el botón de añadir actividad (botón flotante de abajo a la derecha) se nos abre un cuadro de diálogo en el que podemos rellenar cada uno de los campos. Las actividades están prefijadas de antemano, sólo se incorporan aquellas que son más importantes para posteriores estudios.
\newline

Por último, cada actividad cuenta con tres puntitos, los cuales despliegan el menú de opciones. Las opciones disponibles son editar y borrar. Editar abrirá un cuadro de diálogo como el de añadir, pero con todos los campos rellenos con la información de dicha actividad (aparecerán bloqueados aquellos que no se puedan editar) Borrar abrirá un cuadro de diálogo preguntando si de verdad se desea borrar la actividad, en caso afirmativo la actividad será borrada de la lista que contiene todas las actividades.

\subsubsection{Medicamentos}
\begin{figure}[!htb]
\minipage{0.32\textwidth}
  \includegraphics[width=\linewidth]{Medicamentos.jpg}
  \caption{Medicamentos}
\endminipage\hfill
\minipage{0.32\textwidth}
  \includegraphics[width=\linewidth]{AnyadirMedicamento.jpg}
  \caption{Añadir medicamento}
\endminipage\hfill
\minipage{0.32\textwidth}%
  \includegraphics[width=\linewidth]{MenuMedicamentos.jpg}
  \caption{Menú medicamento}
\endminipage
\end{figure}

Las tres pantallas de arriba muestran cómo se trabaja con los medicamentos. La pantalla principal cuenta con una lista de todos los medicamentos que tiene que tomar el paciente y con un botón que permite añadir un nuevo medicamento al sistema.
\newline

La información que se guarda de cada medicamento es:

\begin{itemize}
	\item {\bf Nombre} del medicamento.
	\item {\bf Intervalo} para la recogida de datos. Este parámetro indica los minutos después de la toma del medicamento, en los que los datos recogidos por los sensores podrían verse alterados debido al efecto de la medicación.
	\item {\bf Días de la semana} en los que se tiene que tomar la medicación.
	\item {\bf Hora} a la que se tiene que tomar la medicación.
\end{itemize}

Una vez es añadido el medicamento, la próxima vez que se inicie la aplicación se programarán notificaciones que avisarán al paciente de cuándo se tiene que tomar cada medicamento ese día.
\newline

Al pulsar en el botón de añadir medicamento (botón flotante de abajo a la derecha) se nos abre un cuadro de diálogo en el que podemos rellenar cada uno de los campos.
\newline

Por último, cada medicamento cuenta con tres puntitos, los cuales despliegan el menú de opciones. Las opciones disponibles son editar y borrar. Editar abrirá un cuadro de diálogo como el de añadir, pero con todos los campos rellenos con la información de dicho medicamento (aparecerán bloqueados aquellos que no se puedan editar) Borrar abrirá un cuadro de diálogo preguntando si de verdad se desea borrar el medicamento, en caso afirmativo el medicamento será borrado de la lista que contiene todos los medicamentos.

\subsubsection{Servidor}
\begin{figure}[!htb]
\centering
\includegraphics[width=0.25\textwidth]{EnviarDatos.jpg}
\caption{Enviar datos al servidor.}
\end{figure}

El botón de enviar que se encuentra situado arriba a la derecha, permite enviar todos los datos al servidor. Se enviarán los datos recogidos por los sensores, las actividades realizadas por el paciente y los medicamentos que se tiene que tomar.
\newline

Una vez pulsamos el botón, se abre el cuadro de progreso que puede verse en la imagen. Desaparece una vez todos los datos han sido enviados. Después se informa al usuario de si ha ido todo bien o ha fallado algo.
\newline

\subsubsection{Configuración}
\begin{figure}[!htb]
\centering
\includegraphics[width=0.25\textwidth]{Emparejar_1_.jpg}
\caption{Pantalla de configuración.}
\end{figure}

Cuando pulsamos en el botón emparejar que puede verse en la parte superior derecha (anterior imagen) nos aparece esta pantalla.

En ella se pueden configurar los sensores. Permite indicar el rango del acelerómetro (“es el rango de amplitud máxima que puede medir el acelerómetro antes que la señal de salida resulte distorsionada o recortada. Normalmente se especifica en "g".”)
\newline

El periodo con el que se desean recoger los datos.
Qué sensores se desean activar.
Y en qué parte del cuerpo se va a poner cada uno de los sensores.
\newline

Una vez le demos a comenzar, podremos emparejar el sensor con la aplicación y comenzar a recoger datos.
\newline

\subsubsection{Ayuda}
\begin{figure}[!htb]
\centering
\includegraphics[width=0.25\textwidth]{Ayuda.jpg}
\caption{Ayuda menú inicio.}
\end{figure}

En todas las pantallas de la aplicación puede verse un botón flotante abajo a la izquierda, es el botón de ayuda. Puesto que la aplicación está dirigida a pacientes de Parkinson, en su gran mayoría tienen avanzada edad, consideramos que era buena idea incluir un botón que muestre información acerca de cómo se realiza cada acción.
\newline

El botón de ayuda, muestra en cada pantalla cómo realizar todas las acciones permitidas en dicha pantalla. Como podemos ver en este caso muestra cómo acceder a cada apartado, así como enviar datos y conectar los sensores. Resalta en negrita las partes más importantes, para que puedan ser vistas de forma sencilla y rápida.
\newpage

\subsection{Diagramas UML (Casos de uso)}
\begin{figure}[!htb]
\minipage{0.5\textwidth}
  \includegraphics[width=\linewidth]{AplicacionUML.png}
  \caption{Aplicación}
\endminipage\hfill
\minipage{0.5\textwidth}
  \includegraphics[width=\linewidth]{ActividadesUML.png}
  \caption{Actividades}
\endminipage\hfill
\end{figure}

\begin{figure}[!htb]
\minipage{0.5\textwidth}%
  \includegraphics[width=\linewidth]{MedicamentosUML.png}
  \caption{Medicamentos}
\endminipage
\minipage{0.5\textwidth}%
  \includegraphics[width=\linewidth]{SensoresUML.png}
  \caption{Sensores}
\endminipage
\end{figure}
\newpage

\section{Infraestructura}
\subsection{Sensores}
\subsubsection{Texas Instrument: CC2650 - Ultra-low power wireless MCU}
{\bf Descripción}
\newline

El dispositivo es un sensor BLE (Bluetooth Low Energy) es decir, es bluetooth de bajo consumo. Cuenta con varios sensores dentro del propio dispositivo. Se puede conectar con cualquier dispositivo que disponga de bluetooth, en nuestro caso se ha utilizado conectándolo con un teléfono móvil. 

\begin{figure}[h!]
  \centering
  \includegraphics[width=0.5\textwidth]{SensorTexas.jpg}
  \caption{Sensor CC2650.}
\end{figure}

\begin{figure}[h!]
  \centering
  \includegraphics[width=0.5\textwidth]{FundaTexas.png}
  \caption{Aspecto del sensor con la funda.}
\end{figure}

{\bf Hardware}
\newline

El dispositivo contiene un procesador ARM de 32 bits que funciona a 48 MHz, el cual se encarga de controlar el funcionamiento de todos los sensores que lo forman. Los servicios disponibles son los siguientes:

\begin{itemize}
  \item Sensor de temperatura.
  \item Sensor de movimiento (acelerómetro y giroscopio)
  \item Sensor de humedad.
  \item Sensor de presión.
  \item Sensor óptico.
\end{itemize}

\begin{figure}[h!]
  \centering
  \includegraphics[width=\textwidth]{ArquitecturaTexas.jpg}
  \caption{Arquitectura interna sensor texas.}
\end{figure}

{\bf Software}
\newline

Para interactuar con el sensor se puede utilizar cualquier dispositivo que tenga bluetooth, desde un dispositivo móvil hasta un ordenador. Existen librerías en Python que facilitan la interacción con el mismo. Para acceder a los sensores se utilizan direcciones físicas, conocidas como UUID (Universally Unique Identifier) También hay UUID que permiten configurar el sensor para cambiar parámetros como la frecuencia de funcionamiento, que sensores deben activarse, etc...
\newline

{\bf Uso}
\newline

El sensor lo hemos utilizado con Android. Para ello hemos creado una aplicación móvil que se conecta al sensor y obtiene los datos de él. Para la conexión con el sensor se utiliza la dirección MAC que tiene asociada. Antes de realizar la conexión, se indican que servicios se desean activar y con que frecuencia se quieren recoger datos. Después, se realiza un escaneo y al seleccionar el sensor se empareja con él y comienza a leer datos. Los datos son almacenados en el teléfono y posteriormente son enviados a la nube para su posterior tratamiento.

\subsubsection{Hexiwear}
{\bf Descripción}
\newline

Es un dispositivo pensado para IoT (Internet de las cosas) bastante pequeño y con un diseño amigable que permite usarlo como reloj inteligente. Hexiwear es un sensor bastante versátil completamente software libre.
\newline

\begin{figure}
  \centering
  \includegraphics[width=6cm]{sensorHexiwearCorrea.png}
  \caption{Sensor Hexiwar con correa.}
\end{figure}

{\bf Harwdare}
\newline

El dispositivo cuenta con un MCU Kinetis K64x de NXP (ARM Cortex-M4, 120 MHz, 1M Flash, 256K SRAM), BLE NXP Kinetis KW4x (ARM® Cortex-M0+, Bluetooth Low Energy \& 802.15.4 Wireless MCU), batería de litio, pantalla OLED
de 1.1'', interfaz táctil y un led RGB. Ademas cuenta con los siguientes sensores:

\begin{itemize}
  \item Acelerómetro tridimensional.
  \item Magnetómetro tridimensional.
  \item Giroscopio triaxial.
  \item Sensor de presión.
  \item Sensor de pulsaciones.
  \item Sensor de humedad
  \item Sensor óptico
\end{itemize}

\begin{figure}[H]
  \centering
  \includegraphics[width=15cm]{Hexiwear-hardware-diagram.png}
  \caption{Diagrama del Hardware (Hexiwear).}
\end{figure}

{\bf Software}
\newline

Hexiwear ofrece aplicaciones tanto en iOS como en Android para interactuar con sus sensores, además ofrece una nube a traves de la cual recuperar los datos del sensor haciendo uso de un API REST. Dado que nuestra aplicación esta diseñada para ser capaz de monitorizar a un paciente en cualquier momento no podíamos hacer uso de la nube ya que necesita conexión a Internet. Nuestra aplicación integra la comunicación con el sensor haciendo uso de BLE.
\newline

{\bf Uso}
\newline

Como se menciona en el apartado anterior la interacción con el sensor se hace a través de nuestra propia aplicación Android. Mientras el dispositivo esta conectado con la aplicación se recogen los datos de acelerómetro y giroscopio y se almacenan en la base de datos local de la aplicación para posteriormente enviarlos y procesarlos en el servidor remoto.

\newpage
\begin{thebibliography}{10}
	\bibitem{resumen1}
    Shyamal Patel, Konrad Lorincz, Richard Hughes, Nancy Huggins, John H. Growdon, Matt Welsh, Paolo 		Bonato, Senior Member, IEEE. "Analysis of Feature Space for Monitoring Persons with Parkinson’s
	Disease With Application to a Wireless Wearable Sensor System"
    \bibitem{resumen2}
    Walter Maetzler, Josefa Domingos, Karin Srulijes, Joaquim J. Ferreira, Bastiaan R. Bloem.       	     "Quantitative Wearable Sensors for Objective Assessment of Parkinson’s Disease"
    \bibitem{resumen3}
    BAUKJE DIJKSTRA, WIEBREN ZIJLSTRA, ERIK SCHERDER, YVO KAMSMA. "Detection of walking periods and 	     number of steps in older adults and patients with Parkinson’s disease: accuracy of a pedometer and       an accelerometry-based method"
    \bibitem{resumen4}
    Shyamal Patel, Konrad Lorincz, Richard Hughes, Nancy Huggins, John Growdon, David Standaert, Metin       Akay, Fellow, IEEE, Jennifer Dy, Matt Welsh, Member, IEEE, and Paolo Bonato, Senior Member, IEEE.       "Monitoring Motor Fluctuations in Patients With Parkinson’s Disease Using Wearable Sensors"
    \bibitem{Texas Instruments} http://www.ti.com/product/CC2650/description
    \bibitem{Wiki TI} http://processors.wiki.ti.com/index.php/CC2650\_SensorTag\_User's\_Guide
    \bibitem{App TI} https://play.google.com/store/apps/details?id=com.ti.ble.sensortag\&hl=es
\end{thebibliography}

\end{document}
