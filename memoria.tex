\documentclass[11pt,spanish]{article}
%% Language and font encodings
\usepackage[utf8x]{inputenc}
\usepackage[T1]{fontenc}
\usepackage[spanish,activeacute]{babel}

%% Sets page size and margins
\usepackage[a4paper,top=3cm,bottom=2cm,left=3cm,right=3cm,marginparwidth=1.75cm]{geometry}
\usepackage{indentfirst}

\begin{document}
\section{Estado del arte}

“La enfermedad de Parkinson (EP), también denominada mal de Parkinson, parkinsonismo idiopático, parálisis agitante o simplemente párkinson, es una enfermedad neurodegenerativa crónica caracterizada por bradicinesia (movimiento lento), rigidez (aumento del tono muscular) y temblor.” \cite{Wikipedia} 
\newline

Uno de los artículos que mejor puede servir como introducción para ver cuales son las necesidades, acerca del diagnostico y la supervisor de enfermos de Parkinson usando sensores ``wearables``, y hasta que punto han sido cubiertas o estudiadas es \cite{resumen2}, en este articulo se habla sobre el creciente interés que surge alrededor del estudio de los distintos síntomas del Parkinson usando tecnología ``wearable`` y de como el uso de esta tecnología, permitiría recolectar datos de forma poco intrusiva, objetiva y ecológica para obtener diagnósticos mas precisos y menos sujetos a la subjetividad del paciente o a como este expresa los síntomas sufridos, ya que muchas veces durante las consultas médicas los síntomas no se muestran y el medico no puede evaluarlos por si mismo.
\newline

``Specifically, quantitative assessments using wearable technology may allow for continuous, unobtrusive, objective, and ecologically valid data collection.``
\newline

Ademas el artículo nos hace un recorrido sobre los distintos síntomas del Parkinson explicando como el uso de la tecnología wearable podría ayudar a trabajar sobre cada uno de ellos. El artículo también habla sobre la necesidad de identificar la existencia de sedentarismo y evitarla en pacientes de Parkinson para poder proporcionar una buen nivel de vida.
\newline

Nuestro sistema cubre gran parte las necesidades enumeradas en este artículo, ya que se puede adaptar al estudio de los distintos síntomas del Parkinson enumerados en el articulo dado que esta diseñado para ser un sistema genérico e independiente de la posición de los sensores y del tipo de actividad que se realiza. Incluso cubre la necesidad de identificar el sedentarismo ya que se le puede solicitar al paciente que grave distintas actividades de su vida cotidiana para ver el nivel de actividad física que realiza en su día a día.
\newline

\subsection{Tipos de sensores usados}
A la hora de seleccionar el tipo de sensores a utilizar todos los artículos que hemos leído usan, ya sea en solitario o acompañado por otros tipos de sensores, acelerómetros triaxiales. En \cite{resumen3} realizan una comparación entre los datos recogidos por un acelerómetro y un podómetro para estudiar la congelación de pisada o FOG por sus siglas en inglés, en este articulo hacen que tanto pacientes de parkinson como persona que no padecen esta enfermedad caminen durante longitudes variables y después entrenan algoritmos de aprendizaje automático con estos datos para intentar obtener el numero de pasos que se han realizado. Una de las conclusiones del articulo es que los datos obtenidos por el acelerómetro ofrecen un mejor resultado que los obtenidos por el podómetro para longitudes superiores a 5 metros. 
\newline

Otro artículo en el que se habla sobre el uso de acelerómetros para monitorizar a pacientes de Parkinson es \cite{resumen4}, en este artículo se estudia la factibilidad de usar acelerómetros para identificar la gravedad de discapacidades motoras. En concreto temblores, bradicinesia, y discinesia. Llegando a obtener unas tasas de error bastante bajas, Para ello sitúan sensores en varias partes del cuerpo y recogen los datos. Estos datos se usan después para entrenar un algoritmo de aprendizaje automático (Support Vector Machine) cuyo resultado se compara con el diagnostico realzado por un medico. Una de las limitaciones que vemos en este artículo, y que no tendremos en este proyecto, es el uso de sensores alámbricos que son mucho mas intrusivos para el paciente que los lleva.

Tras la lectura de los artículos \cite{resumen3} y \cite{resumen4} entre otros, podemos concluir que el uso de acelerómetros como sensores para realizar este proyecto es bastante acertado ya que en la mayoría de estudios obtienen resultados satisfactorios. Además, en nuestro proyecto, usaremos giroscopios para apoyar la tarea de los acelerómetros e intentar mejorar los resultados.

\subsection{Síntomas del Parkinson estudiados}
Como se cuenta en \cite{resumen2} el Parkinson tiene diferentes síntomas, algunos motores, la gran mayoría, y algunos no motores. 
\newline

En este proyecto nos centraremos en los síntomas motores, la mayoría han sido estudiados ya de alguna forma, por ejemplo en \cite{resumen3} se centran en el estudio de la congelación de pisada \cite{gait}, \cite{resumen5} se centra en el estudio de los temblores o \cite{resumen4} en el que se estudian varios sintomas, en este en concreto se estudian temblores, bradicinesia \cite{hipocinesia} y discinesia \cite{dyskinesia}.
\newline

Nuestro sistema puede usarse para recopilar datos de todos los síntomas motores que tiene el Parkinson ya que las diferencias entre estos suele en las posiciones seleccionadas para poner los sensores o en las actividades que realiza el paciente mientras que se le esta monitorizando y nuestro sistema ofrece una libertad total respecto a estas opciones.

\subsection{Libertad vs Actividades}

En \cite{resumen2} se habla sobre la monitorización continua y no intrusiva de pacientes Parkinson, aun así, en la gran mayoría de los artículos que hemos leído la monitorización de enfermos de parkinson, ya sea en un entorno controlado o no, suele ir ligada a la realización repetitiva de distintas actividades que ayudan a caracterizar los síntomas que se estén estudiando en cada caso. Nosotros pensamos que la realización de estas tareas en si, pese a que aportan información muy valiosa sobre como evoluciona el paciente, son intrusivas ya que obligas al paciente a realizarlas de forma periódica y es necesario contar con la colaboración de este. Por este motivo uno de los objetivos a largo plazo es que se pueda monitorizar al paciente realizando tareas del día a día con total libertad.
\newline

Nuestro sistema tiene en mente estas necesidades y esta diseñado para ser suficientemente genérico y permitir tanto la grabación del paciente realizando tares establecidas por un medico o el encargado del estudio de este como tareas de su día a día.

\subsection{Innovación}

Una de las características que tiene nuestro sistema y la cual no hemos visto en ninguno de los artículos leídos, es que nuestros sistema no solo se encarga de monitorizar y almacenar los datos si no que también es capaz de enviarlos a un servidor remoto y ofrece una interfaz, en forma de aplicación web, para que el encargado de analizar los datos, normalmente el médico pueda analizarlos y trabajar sobre ellos de forma remota sin tener que estar con el paciente mientras este realiza los datos o tener que concertar citas para extraer los datos del sistemas pertinente.
\newline

Creemos que esta funcionalidad cierra el ciclo del sistema y lo hace totalmente independiente y e idóneo para la tarea para la que esta diseñado, ademas nuestro sistema ofrece mucha facilidad en caso de querer añadir funcionalidades, como por ejemplo aplicar algoritmos de aprendizaje automático sobre los datos ya que siempre se puede acceder a los datos haciendo uso del API REST y a partir de ahí implementar la nueva funcionalidad.


\end{document}
